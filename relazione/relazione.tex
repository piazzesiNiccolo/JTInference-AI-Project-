\documentclass{report}
\usepackage[utf8]{inputenc}
\usepackage[T1]{fontenc}
\usepackage[italian]{babel}
\usepackage{makecell}
\usepackage{threeparttable}


\renewcommand\theadalign{bc}
\renewcommand\theadfont{\bfseries}
\renewcommand\theadgape{\Gape[4pt]}
\renewcommand\cellgape{\Gape[4pt]}

\begin{document}
\title{ \textbf{Inferenza con junction trees nei belief networks}}
\author{Niccolò Piazzesi\\ \texttt{Anno Accademico 2019/20}}
\date{}
\maketitle

\section*{Introduzione}
Il progetto è diviso in due sottocartelle:
\begin{itemize}
\item in \textbf{tree} si trova il codice per la costruzione dei junction tree e l'algoritmo di belief propagation
\item in \textbf{examples} si trovano gli esempi costruiti a partire da  reti presenti nella cartella samples di Hugin
\end{itemize}

\noindent Le reti utilizzate sono state  incluse anche nella cartella \textbf{hugin\_networks}.\\
In ogni esempio viene eseguita una query su ciascuna variabile aleatoria presente nella rete, mostrando  la sua probabilità marginale  prima e dopo aver inserito evidenza. La verifica numerica  è stata fatta eseguendo la stessa query all'interno di Hugin, controllando che i risultati coincidessero(al netto di piccoli errori di approssimazione). \\\\
\noindent Per eseguire un determinato esempio, avviare \textbf{run\_examples.py} e selezionarlo dal prompt che compare. Ogni esempio può essere rieseguito più volte.\\

\noindent Nella sezione \textbf{Risultati} vengono tabulati i dati ottenuti, divisi per ciascun esempio. In ogni tabella vengono mostrate le query eseguite e i risultati ottenuti sia nel programma che in Hugin.
\newpage
\section*{Risultati}
\paragraph{N.B} 
Nelle tabelle le  probabilità marginali sono rappresentate come tuple. Se ad esempio ho una variabile binaria v con stati \{no, yes\} p(v) sarà scritta come (v $=$ no, v$=$yes).\\\\
\begin{table}[tbh!]
\small
\begin{tabular}{ |p{2cm}||p{2cm}|p{2cm}|p{3cm}|p{3cm}|  }
\hline
 \multicolumn{5}{|c|}{Icy Roads} \\
 \hline
 Variabile & P(var) & Evidenza & P(var | evidenza) & P(var | evidenza)  in Hugin\\
 \hline
 Icy & (0.3, 0.7) & Holmes$=$no, Watson$=$yes & (0.1942, 0.8058) & (0.1942, 0.8058)\\
\hline
 Watson & (0.41, 0.59) & Holmes$=$yes & (0.2356, 0.7644) & (0.2356, 0.7644)\\
\hline
 Holmes & (0.41, 0.59) & Icy$=$no & (0.9, 0.1) & (0.9, 0.1)\\
\hline
\end{tabular}
\newline
\vspace*{0.5 cm}
\newline

\begin{tabular}{ |p{2cm}|p{2cm}|p{2cm}|p{3cm}|p{3cm}|  }
\hline
 \multicolumn{5}{|c|}{Mrs Gibbon} \\
 \hline
Variabile & P(var) & Evidenza & P(var | evidenza) & P(var | evidenza)  in Hugin\\
 \hline
Holmes &(0.8199,0.1801) & Watson$=$yes, Sprinkler$=$yes & (0.0476, 0.9524) &(0.0476, 0.9524)\\
\hline
Watson & (0.811,0.189) & Gibbon $=$ yes, Holmes $=$ no & (0.8904,1096)& (0.8904,1096)\\
\hline
Rain & (0.9,0.1) & Watson$=$yes, Holmes$=$yes & (0.0763, 0.9237) & (0.0763, 0.9237)\\
\hline
Sprinkler & (0.9,0.1) & Holmes$=$yes & (0.4947,0.5053) & (0.4947,0.5053)\\
\hline
Gibbon & (0.811,0.189) & Watson $=$ yes & (0.4338,0.5662) & (0.4338,0.5662)\\
\hline
\end{tabular}
\end{table}
\newpage

\begin{table}[hbt!]
\small

\hskip-1.0cm\begin{tabular}{ |p{2cm}|p{2cm}|p{2cm}|p{3cm}|p{3cm}|  }
\hline
 \multicolumn{5}{|c|}{Fire} \\
 \hline
Variabile & P(var) & Evidenza & P(var | evidenza) & P(var | evidenza)  in Hugin\\
 \hline
Fire &(0.99,0.01) & Leaving$=$yes, Alarm$=$yes & (0.6333, 0.3667) &(0.6333, 0.3667)\\
\hline
Tampering & (0.98,0.02) & Report $=$ yes, Smoke $=$ no & (0.4992,5008)&  (0.4992,5008)\\
\hline
Smoke & (0.9811,0.0189) & Fire$=$yes, Leaving$=$yes & (0.1, 0.9) & (0.1, 0.9)\\
\hline
Alarm & (0.9733,0.0267) & Report$=$yes, Smoke$=$no & (0.4687,0.5313) & (0.4687,0.5313)\\
\hline
Leaving & (0.9755,0.0245) & \makecell[l]{Report$=$no, \\ Fire$=$yes, \\Tampering$=$yes} & (0.3468,0.6532) & (0.3468,0.6532)\\
\hline
Report & (0.9719,0.0281)&Tampering$=$no, Smoke$=$Yes & (0.6798,0.3202) & (0.6798,0.3202)\\
\hline
\end{tabular}
\newline
\vspace*{0.5 cm}
\newline

\hskip-1.0cm\begin{threeparttable}
\begin{tabular}{ |p{2cm}|p{2cm}|p{2cm}|p{3cm}|p{3cm}|  }
\hline
 \multicolumn{5}{|c|}{Cancer Neapolitan} \\
 \hline
Variabile & P(var) & Evidenza & P(var | evidenza) & P(var | evidenza)  in Hugin\\
 \hline
MC &(0.8,0.2) & \makecell[l]{BT$=$yes, \\Coma$=$yes} & (0.6333, 0.3667) &(0.6333, 0.3667)\\
\hline
SC & (0.68,0.32) & MC$=$no & (0.8,0.2)&  (0.8,0.2)\\
\hline
BT & (0.92,0.008) & MC$=$yes, SH$=$yes & (0.8961, 0.1039) & (0.8961, 0.1039)\\
\hline
Coma & (0.656,0.344) & BT$=$yes, SH$=$yes & (0.25,0.75) & (0.25,0.75)\\
\hline
SH & (0.384,0.6160) & BT$=$yes, MC$=$no & (0.2,0.8) &(0.2,0.8)\\
\hline
\end{tabular}
\begin{tablenotes}
 	    \item[1] MC $=$ Metastatic Cancer
        \item[2] BT $=$ Brain Tumor
        \item[3] SC $=$ Serum Calcium
        \item[4] SH $=$ Severe Headaches
 \end{tablenotes}
\end{threeparttable}
\end{table}

\newpage
\begin{table}[hbt!]
\small
\hskip-1.0cm\begin{threeparttable}
\begin{tabular}{ |p{2cm}|p{3cm}|p{2cm}|p{3cm}|p{3cm}|  }
\hline
 \multicolumn{5}{|c|}{Flood} \\
 \hline
Variabile & P(var) & Evidenza & P(var | evidenza) & P(var | evidenza)  in Hugin\\
 \hline
Rain &(0.99,0.01) & Alarm$=$yes & (0.9892, 0.108) &(0.9892, 0.108)\\
\hline
Burglary & (0.5,0.5) & Alarm$=$yes, S$=$no & (0.8221,0.1779)& (0.8221,0.1779)\\
\hline
E & (0.9,0.1) & S$=$1, Burglary $=$ no & (0.1565,0.8435) & (0.1565,0.8435)\\
\hline
Flood & (0.99,0.1) & Alarm$=$yes, Flood$=$no & (0.8324, 0.1676) & (0.8324, 0.1676)\\
\hline
Alarm & (0.4503,0.5497) & E$=$yes& (0.005,0.995) & (0.005,0.995)\\
\hline
S & (0.4415,0.068,0.4905) & Burglary$=$no, E$=$no & (0.49,0.02,0.49) &(0.49,0.02,0.49)\\
\hline
\end{tabular}
\begin{tablenotes}
 	    \item[1] E$=$ Earthquake 
        \item[2] S$=$ Seismometer
        \item[3] p(alarm) in Hugin ha valori(0.4501,0.5499). Essendo l'errore abbastanza piccolo e non influente sui calcoli successivi è stato scelto di ignorarlo
 \end{tablenotes}
\end{threeparttable}
\end{table}


\end{document}